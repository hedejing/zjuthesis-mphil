\chapter{总结与展望}

\section{全文总结}

本文提出了一种\stc{}的轮廓线的实时绘制方法,该方法的核心想法在于通过一个图像空间的搜索来完成轮廓点沿着对极曲线的连续性的判定,而不是像之前的工作中那样对多个视点进行采样来完成连续性的判定。具体地,本文拓展了\epsl{}的概念并推导了一个新的\epsl{}的判定方法:通过轮廓点的对应视点的轨迹的单调性来判定\epsl{}。在这个推导的基础上,本文提出了一个多阶段的绘制算法,该算法首先计算出轮廓线以及轨迹函数的极值点与区间端点,然后根据这些信息在图像空间完成\epsl{}的判定。本文提出的算法同时支持\con{}以及\scon{},并且支持进一步的风格化绘制。实验结果表明,本文提出的方法在正确地消去非\stc{}轮廓线的同时高质量地保留了\stc{}的轮廓线。本文提出的算法对GPU十分友好,所需的所有步骤都可以在着色器中实现。由于本文提出的算法是完全以每一帧的信息作为输入进行计算的,不依赖于任何预计算的信息,所以能够让用户交互式地调整三维模型并且对各种绘制参数进行调整。

\section{未来展望}

尽管本文提出的方法能够实现实时的\stc{}轮廓线绘制,但还该方法还存在一些限制。首先,对于错误匹配的处理还不够完善。在错误匹配来自于同一个物体时,使用物体的索引来排除错误匹配的方法会失效。一个更可靠的用于排除错误匹配的方法还有待开发。其次,在本文的讨论中还没有对轮廓线的时序一致性(temporal coherency)进行考虑。换言之,每帧绘制的轮廓线在时间上可能会不够连续,因此在视频中会出现轮廓线的闪烁。在启用轮廓线风格化的情况下,这种闪烁的情况会更加明显。出现这个问题的根源在于立体一致性和时序一致性之间会存在冲突,如果想要保证立体一致性,则需要根据物体的几何特征进行轮廓线的风格化,而如果想要保证时序一致性,那么需要传递风格化参数,二者如何平衡是一个很难解决的问题。因此,设计出一个能够支持时序一致性的方法来完成\stc{}轮廓线的绘制,是一个很有价值的研究目标。