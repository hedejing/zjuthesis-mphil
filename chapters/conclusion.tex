\chapter{总结与展望}

\section{全文总结}

线绘制是一种表现物体外形的重要技术。按照视点相关性,线绘制技术中探讨的线条可以分为视点相关线和视点无关线。其中,\con{}和\scon{}都属于视点相关线,如果这类线条不以立体一致的方式进行绘制,将会给用户带来不适。针对\vdl{}在双目绘制下出现的问题,本文先是介绍了\vdl{}和双目绘制的概念,然后对前人提出的对极滑动性的概念和立体一致的\vdl{}的绘制方法进行了说明。接着,在前人工作的基础上,本文提出了一种\stc{}\vdl{}的实时绘制方法,该方法的核心想法在于通过一个图像空间的搜索来完成\vdp{}的\epsl{}的判定,而不是像之前的工作中那样对多个视点进行采样来完成\epsl{}判定。具体地,本文拓展了\epsl{}的概念并推导了一个新的\epsl{}的判定方法:通过\vdp{}的对应视点的轨迹的单调性来判定\epsl{}。在这个推导的基础上,本文提出了一个多阶段的绘制算法,该算法首先计算出\vdl{}以及轨迹函数的极值点与区间端点,然后根据这些信息在图像空间完成\epsl{}的判定。本文提出的算法同时支持\con{}以及\scon{},并且支持进一步的风格化绘制。对于\con{}和\scon{}以外的\vdl{},也可以通过同样的过程进行推导,并在同样的算法框架下完成\stc{}的绘制。实验结果表明,本文提出的方法在正确地消去非\stc{}\vdl{}的同时高质量地保留了\stc{}\epsl{}。本文提出的算法对GPU十分友好,其中所有步骤都可以在着色器中实现。由于本文提出的算法是完全以每一帧的信息作为输入进行计算的,不依赖于任何预计算的信息,而且效率很高,所以能够让用户实时地调整三维模型并且对各种绘制参数进行调整。

\section{未来展望}

尽管本文提出的方法能够实现实时的\stc{}\vdl{}绘制,但还该方法还存在一些限制。首先,对于错误匹配的处理还不够完善。在错误匹配来自于同一个物体时,使用物体的索引来排除错误匹配的方法会失效。一个更可靠的用于排除错误匹配的方法还有待开发。其次,在本文的讨论中还没有对\vdl{}的时序一致性进行考虑。换言之,每帧绘制的\vdl{}在时间上可能会不够连续,因此在视频中会出现\vdl{}的闪烁。在启用线条风格化的情况下,这种闪烁的情况会更加明显。出现这个问题的根源在于立体一致性和时序一致性之间会存在冲突,如果想要保证立体一致性,则需要根据物体的几何特征进行线条的风格化,而如果想要保证时序一致性,那么需要传递风格化参数,二者如何平衡是一个很难解决的问题。因此,设计出一个能够支持时序一致性的方法来完成\stc{}\vdl{}的绘制,是一个很有价值的研究方向。